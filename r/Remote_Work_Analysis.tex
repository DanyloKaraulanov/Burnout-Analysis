% Options for packages loaded elsewhere
\PassOptionsToPackage{unicode}{hyperref}
\PassOptionsToPackage{hyphens}{url}
%
\documentclass[
]{article}
\usepackage{amsmath,amssymb}
\usepackage{iftex}
\ifPDFTeX
  \usepackage[T1]{fontenc}
  \usepackage[utf8]{inputenc}
  \usepackage{textcomp} % provide euro and other symbols
\else % if luatex or xetex
  \usepackage{unicode-math} % this also loads fontspec
  \defaultfontfeatures{Scale=MatchLowercase}
  \defaultfontfeatures[\rmfamily]{Ligatures=TeX,Scale=1}
\fi
\usepackage{lmodern}
\ifPDFTeX\else
  % xetex/luatex font selection
\fi
% Use upquote if available, for straight quotes in verbatim environments
\IfFileExists{upquote.sty}{\usepackage{upquote}}{}
\IfFileExists{microtype.sty}{% use microtype if available
  \usepackage[]{microtype}
  \UseMicrotypeSet[protrusion]{basicmath} % disable protrusion for tt fonts
}{}
\makeatletter
\@ifundefined{KOMAClassName}{% if non-KOMA class
  \IfFileExists{parskip.sty}{%
    \usepackage{parskip}
  }{% else
    \setlength{\parindent}{0pt}
    \setlength{\parskip}{6pt plus 2pt minus 1pt}}
}{% if KOMA class
  \KOMAoptions{parskip=half}}
\makeatother
\usepackage{xcolor}
\usepackage[margin=1in]{geometry}
\usepackage{color}
\usepackage{fancyvrb}
\newcommand{\VerbBar}{|}
\newcommand{\VERB}{\Verb[commandchars=\\\{\}]}
\DefineVerbatimEnvironment{Highlighting}{Verbatim}{commandchars=\\\{\}}
% Add ',fontsize=\small' for more characters per line
\usepackage{framed}
\definecolor{shadecolor}{RGB}{248,248,248}
\newenvironment{Shaded}{\begin{snugshade}}{\end{snugshade}}
\newcommand{\AlertTok}[1]{\textcolor[rgb]{0.94,0.16,0.16}{#1}}
\newcommand{\AnnotationTok}[1]{\textcolor[rgb]{0.56,0.35,0.01}{\textbf{\textit{#1}}}}
\newcommand{\AttributeTok}[1]{\textcolor[rgb]{0.13,0.29,0.53}{#1}}
\newcommand{\BaseNTok}[1]{\textcolor[rgb]{0.00,0.00,0.81}{#1}}
\newcommand{\BuiltInTok}[1]{#1}
\newcommand{\CharTok}[1]{\textcolor[rgb]{0.31,0.60,0.02}{#1}}
\newcommand{\CommentTok}[1]{\textcolor[rgb]{0.56,0.35,0.01}{\textit{#1}}}
\newcommand{\CommentVarTok}[1]{\textcolor[rgb]{0.56,0.35,0.01}{\textbf{\textit{#1}}}}
\newcommand{\ConstantTok}[1]{\textcolor[rgb]{0.56,0.35,0.01}{#1}}
\newcommand{\ControlFlowTok}[1]{\textcolor[rgb]{0.13,0.29,0.53}{\textbf{#1}}}
\newcommand{\DataTypeTok}[1]{\textcolor[rgb]{0.13,0.29,0.53}{#1}}
\newcommand{\DecValTok}[1]{\textcolor[rgb]{0.00,0.00,0.81}{#1}}
\newcommand{\DocumentationTok}[1]{\textcolor[rgb]{0.56,0.35,0.01}{\textbf{\textit{#1}}}}
\newcommand{\ErrorTok}[1]{\textcolor[rgb]{0.64,0.00,0.00}{\textbf{#1}}}
\newcommand{\ExtensionTok}[1]{#1}
\newcommand{\FloatTok}[1]{\textcolor[rgb]{0.00,0.00,0.81}{#1}}
\newcommand{\FunctionTok}[1]{\textcolor[rgb]{0.13,0.29,0.53}{\textbf{#1}}}
\newcommand{\ImportTok}[1]{#1}
\newcommand{\InformationTok}[1]{\textcolor[rgb]{0.56,0.35,0.01}{\textbf{\textit{#1}}}}
\newcommand{\KeywordTok}[1]{\textcolor[rgb]{0.13,0.29,0.53}{\textbf{#1}}}
\newcommand{\NormalTok}[1]{#1}
\newcommand{\OperatorTok}[1]{\textcolor[rgb]{0.81,0.36,0.00}{\textbf{#1}}}
\newcommand{\OtherTok}[1]{\textcolor[rgb]{0.56,0.35,0.01}{#1}}
\newcommand{\PreprocessorTok}[1]{\textcolor[rgb]{0.56,0.35,0.01}{\textit{#1}}}
\newcommand{\RegionMarkerTok}[1]{#1}
\newcommand{\SpecialCharTok}[1]{\textcolor[rgb]{0.81,0.36,0.00}{\textbf{#1}}}
\newcommand{\SpecialStringTok}[1]{\textcolor[rgb]{0.31,0.60,0.02}{#1}}
\newcommand{\StringTok}[1]{\textcolor[rgb]{0.31,0.60,0.02}{#1}}
\newcommand{\VariableTok}[1]{\textcolor[rgb]{0.00,0.00,0.00}{#1}}
\newcommand{\VerbatimStringTok}[1]{\textcolor[rgb]{0.31,0.60,0.02}{#1}}
\newcommand{\WarningTok}[1]{\textcolor[rgb]{0.56,0.35,0.01}{\textbf{\textit{#1}}}}
\usepackage{graphicx}
\makeatletter
\newsavebox\pandoc@box
\newcommand*\pandocbounded[1]{% scales image to fit in text height/width
  \sbox\pandoc@box{#1}%
  \Gscale@div\@tempa{\textheight}{\dimexpr\ht\pandoc@box+\dp\pandoc@box\relax}%
  \Gscale@div\@tempb{\linewidth}{\wd\pandoc@box}%
  \ifdim\@tempb\p@<\@tempa\p@\let\@tempa\@tempb\fi% select the smaller of both
  \ifdim\@tempa\p@<\p@\scalebox{\@tempa}{\usebox\pandoc@box}%
  \else\usebox{\pandoc@box}%
  \fi%
}
% Set default figure placement to htbp
\def\fps@figure{htbp}
\makeatother
\setlength{\emergencystretch}{3em} % prevent overfull lines
\providecommand{\tightlist}{%
  \setlength{\itemsep}{0pt}\setlength{\parskip}{0pt}}
\setcounter{secnumdepth}{-\maxdimen} % remove section numbering
\usepackage{bookmark}
\IfFileExists{xurl.sty}{\usepackage{xurl}}{} % add URL line breaks if available
\urlstyle{same}
\hypersetup{
  pdftitle={Remote Work Health Impact Analysis},
  pdfauthor={Danylo Karaulanov},
  hidelinks,
  pdfcreator={LaTeX via pandoc}}

\title{Remote Work Health Impact Analysis}
\author{Danylo Karaulanov}
\date{2025-08-15}

\begin{document}
\maketitle

\begin{verbatim}
## # A tibble: 6 x 14
##   Survey_Date   Age Gender Region        Industry      Job_Role Work_Arrangement
##   <date>      <dbl> <chr>  <chr>         <chr>         <chr>    <chr>           
## 1 2025-06-01     27 Female Asia          Professional~ Data An~ Onsite          
## 2 2025-06-01     37 Female Asia          Professional~ Data An~ Onsite          
## 3 2025-06-01     32 Female Africa        Education     Busines~ Onsite          
## 4 2025-06-01     40 Female Europe        Education     Data An~ Onsite          
## 5 2025-06-01     30 Male   South America Manufacturing DevOps ~ Hybrid          
## 6 2025-06-01     52 Male   Oceania       Customer Ser~ Busines~ Onsite          
## # i 7 more variables: Hours_Per_Week <dbl>, Mental_Health_Status <chr>,
## #   Burnout_Level <chr>, Work_Life_Balance_Score <dbl>,
## #   Physical_Health_Issues <chr>, Social_Isolation_Score <dbl>,
## #   Salary_Range <chr>
\end{verbatim}

\section{\texorpdfstring{\textbf{Introduction}}{Introduction}}\label{introduction}

This report analyzes how different work arrangements (remote, hybrid,
and onsite) affect employee health and burnout based on survey data from
over 3,000 participants.

The analysis focuses on the following research questions:

\begin{itemize}
\tightlist
\item
  Does work arrangement influence burnout levels? - How do gender, age,
  and mental health status correlate with burnout? - What practical
  steps can organizations take to address identified risks?
\end{itemize}

\section{\texorpdfstring{\textbf{Key
Findings}}{Key Findings}}\label{key-findings}

\subsection{\texorpdfstring{\textbf{1. Burnout by
Gender}}{1. Burnout by Gender}}\label{burnout-by-gender}

Analysis shows variation in burnout levels across genders. For example,
female employees reported slightly higher `High Burnout' percentages in
remote and onsite settings compared to males. Charts from R analysis
provide visual confirmation.

\begin{Shaded}
\begin{Highlighting}[]
\NormalTok{burnout\_by\_gender }\OtherTok{\textless{}{-}}\NormalTok{ data }\SpecialCharTok{\%\textgreater{}\%}
  \FunctionTok{count}\NormalTok{(Gender, Work\_Arrangement, Burnout\_Level) }\SpecialCharTok{\%\textgreater{}\%}
  \FunctionTok{group\_by}\NormalTok{(Gender, Work\_Arrangement) }\SpecialCharTok{\%\textgreater{}\%}
  \FunctionTok{mutate}\NormalTok{(}\AttributeTok{Percentage =}\NormalTok{ n }\SpecialCharTok{/} \FunctionTok{sum}\NormalTok{(n) }\SpecialCharTok{*} \DecValTok{100}\NormalTok{) }\SpecialCharTok{\%\textgreater{}\%}
  \FunctionTok{ungroup}\NormalTok{()}
\FunctionTok{head}\NormalTok{(burnout\_by\_gender)}
\end{Highlighting}
\end{Shaded}

\begin{verbatim}
## # A tibble: 6 x 5
##   Gender Work_Arrangement Burnout_Level     n Percentage
##   <chr>  <chr>            <chr>         <int>      <dbl>
## 1 Female Hybrid           High            170       37.4
## 2 Female Hybrid           Low              79       17.4
## 3 Female Hybrid           Medium          205       45.2
## 4 Female Onsite           High            212       27.5
## 5 Female Onsite           Low             230       29.8
## 6 Female Onsite           Medium          330       42.7
\end{verbatim}

\begin{Shaded}
\begin{Highlighting}[]
\FunctionTok{ggplot}\NormalTok{(burnout\_by\_gender, }\FunctionTok{aes}\NormalTok{(}\AttributeTok{x =}\NormalTok{ Work\_Arrangement,}
                              \AttributeTok{y =}\NormalTok{ Percentage,}
                              \AttributeTok{fill =}\NormalTok{ Burnout\_Level)) }\SpecialCharTok{+}
  \FunctionTok{geom\_col}\NormalTok{(}\AttributeTok{position =} \StringTok{"dodge"}\NormalTok{) }\SpecialCharTok{+} 
  \FunctionTok{facet\_wrap}\NormalTok{(}\SpecialCharTok{\textasciitilde{}}\NormalTok{Gender) }\SpecialCharTok{+} 
  \FunctionTok{labs}\NormalTok{(}
    \AttributeTok{title =} \StringTok{"Burnout Levels by Work Arrangement and Gender"}\NormalTok{,}
    \AttributeTok{x =} \StringTok{"Work Arrangement"}\NormalTok{,}
    \AttributeTok{y =} \StringTok{"Percentage (\%)"}\NormalTok{,}
    \AttributeTok{fill =} \StringTok{"Burnout Level"}
\NormalTok{  ) }\SpecialCharTok{+} 
  \FunctionTok{theme\_minimal}\NormalTok{() }\SpecialCharTok{+} 
  \FunctionTok{scale\_fill\_brewer}\NormalTok{(}\AttributeTok{palette =} \StringTok{"RdYlGn"}\NormalTok{, }\AttributeTok{direction =} \SpecialCharTok{{-}}\DecValTok{1}\NormalTok{) }\SpecialCharTok{+}
  \FunctionTok{theme}\NormalTok{(}\AttributeTok{axis.text.x =} \FunctionTok{element\_text}\NormalTok{(}\AttributeTok{angle =} \DecValTok{45}\NormalTok{, }\AttributeTok{hjust =} \DecValTok{1}\NormalTok{))}
\end{Highlighting}
\end{Shaded}

\pandocbounded{\includegraphics[keepaspectratio]{Remote_Work_Analysis_files/figure-latex/unnamed-chunk-2-1.pdf}}

\subsection{\texorpdfstring{\textbf{2. Burnout by Work
Arrangement}}{2. Burnout by Work Arrangement}}\label{burnout-by-work-arrangement}

Remote workers consistently reported higher burnout than hybrid and
onsite employees. Hybrid workers had the lowest proportion of
high-burnout cases.

\begin{Shaded}
\begin{Highlighting}[]
\NormalTok{burnout\_by\_arrangement }\OtherTok{\textless{}{-}}\NormalTok{ data }\SpecialCharTok{\%\textgreater{}\%}
  \FunctionTok{group\_by}\NormalTok{(Work\_Arrangement, Burnout\_Level) }\SpecialCharTok{\%\textgreater{}\%}
  \FunctionTok{summarise}\NormalTok{(}\AttributeTok{Count =} \FunctionTok{n}\NormalTok{()) }\SpecialCharTok{\%\textgreater{}\%}
  \FunctionTok{mutate}\NormalTok{(}\AttributeTok{Percentage =}\NormalTok{ (Count }\SpecialCharTok{/} \FunctionTok{sum}\NormalTok{(Count)) }\SpecialCharTok{*} \DecValTok{100}\NormalTok{)}
\FunctionTok{head}\NormalTok{(burnout\_by\_arrangement)}
\end{Highlighting}
\end{Shaded}

\begin{verbatim}
## # A tibble: 6 x 4
## # Groups:   Work_Arrangement [2]
##   Work_Arrangement Burnout_Level Count Percentage
##   <chr>            <chr>         <int>      <dbl>
## 1 Hybrid           High            360       35.7
## 2 Hybrid           Low             193       19.2
## 3 Hybrid           Medium          454       45.1
## 4 Onsite           High            413       26.4
## 5 Onsite           Low             471       30.2
## 6 Onsite           Medium          678       43.4
\end{verbatim}

\begin{Shaded}
\begin{Highlighting}[]
\FunctionTok{ggplot}\NormalTok{(burnout\_by\_arrangement,}
       \FunctionTok{aes}\NormalTok{(}\AttributeTok{x =}\NormalTok{ Work\_Arrangement, }\AttributeTok{y =}\NormalTok{ Percentage, }\AttributeTok{fill =}\NormalTok{ Burnout\_Level)) }\SpecialCharTok{+}
  \FunctionTok{geom\_col}\NormalTok{(}\AttributeTok{width =} \FloatTok{0.6}\NormalTok{) }\SpecialCharTok{+} 
  \FunctionTok{scale\_fill\_manual}\NormalTok{(}\AttributeTok{values =} \FunctionTok{c}\NormalTok{(}\StringTok{"Low"} \OtherTok{=} \StringTok{"\#A6D8A0"}\NormalTok{, }\StringTok{"Medium"} \OtherTok{=} \StringTok{"\#F9CB8F"}\NormalTok{, }\StringTok{"High"} \OtherTok{=} \StringTok{"\#F38D8D"}\NormalTok{)) }\SpecialCharTok{+} 
  \FunctionTok{labs}\NormalTok{(}
    \AttributeTok{x =} \StringTok{"Work Type"}\NormalTok{,}
    \AttributeTok{y =} \StringTok{"Percentage of Employees"}\NormalTok{,}
    \AttributeTok{fill =} \StringTok{"Burnout Level"}
\NormalTok{  ) }\SpecialCharTok{+} 
  \FunctionTok{theme\_minimal}\NormalTok{() }\SpecialCharTok{+} 
  \FunctionTok{theme}\NormalTok{(}
    \AttributeTok{panel.grid =} \FunctionTok{element\_blank}\NormalTok{(),}
    \AttributeTok{axis.text =} \FunctionTok{element\_text}\NormalTok{(}\AttributeTok{size =} \DecValTok{11}\NormalTok{), }
    \AttributeTok{legend.position =} \StringTok{"top"}
\NormalTok{  )}
\end{Highlighting}
\end{Shaded}

\pandocbounded{\includegraphics[keepaspectratio]{Remote_Work_Analysis_files/figure-latex/unnamed-chunk-3-1.pdf}}

\subsection{\texorpdfstring{\textbf{3. Mental Health and
Burnout}}{3. Mental Health and Burnout}}\label{mental-health-and-burnout}

Employees with pre-existing mental health conditions (such as anxiety,
ADHD, or stress disorders) showed a much higher likelihood of reporting
`High Burnout.' Visualization confirms the strong overlap between mental
health challenges and burnout.

\begin{verbatim}
## # A tibble: 6 x 4
## # Groups:   Mental_Health_Status [2]
##   Mental_Health_Status Burnout_Level Count Percentage
##   <chr>                <chr>         <int>      <dbl>
## 1 ADHD                 High            119       30.9
## 2 ADHD                 Low             109       28.3
## 3 ADHD                 Medium          157       40.8
## 4 Anxiety              High            136       34.5
## 5 Anxiety              Low              92       23.4
## 6 Anxiety              Medium          166       42.1
\end{verbatim}

\begin{Shaded}
\begin{Highlighting}[]
\FunctionTok{ggplot}\NormalTok{(mental\_health\_burnout, }\FunctionTok{aes}\NormalTok{(}\AttributeTok{x =}\NormalTok{ Mental\_Health\_Status, }\AttributeTok{y =}\NormalTok{ Percentage, }\AttributeTok{fill =}\NormalTok{ Burnout\_Level)) }\SpecialCharTok{+} 
  \FunctionTok{geom\_bar}\NormalTok{(}\AttributeTok{stat =} \StringTok{"identity"}\NormalTok{, }\AttributeTok{position =} \StringTok{"dodge"}\NormalTok{) }\SpecialCharTok{+} 
  \FunctionTok{labs}\NormalTok{(}\AttributeTok{title =} \StringTok{"Burnout Levels by Mental Health Status"}\NormalTok{, }\AttributeTok{x =} \StringTok{"Mental Health Condition"}\NormalTok{, }\AttributeTok{y =} \StringTok{"Percentage (\%)"}\NormalTok{) }\SpecialCharTok{+}
  \FunctionTok{theme\_minimal}\NormalTok{() }\SpecialCharTok{+}
  \FunctionTok{theme}\NormalTok{(}\AttributeTok{axis.text.x =} \FunctionTok{element\_text}\NormalTok{(}\AttributeTok{angle =} \DecValTok{45}\NormalTok{, }\AttributeTok{hjust =} \DecValTok{1}\NormalTok{))}
\end{Highlighting}
\end{Shaded}

\pandocbounded{\includegraphics[keepaspectratio]{Remote_Work_Analysis_files/figure-latex/unnamed-chunk-4-1.pdf}}

\begin{Shaded}
\begin{Highlighting}[]
\NormalTok{data }\OtherTok{\textless{}{-}}\NormalTok{ data }\SpecialCharTok{\%\textgreater{}\%} 
  \FunctionTok{mutate}\NormalTok{(}\AttributeTok{Age\_Group =} \FunctionTok{case\_when}\NormalTok{(}
\NormalTok{    Age }\SpecialCharTok{\textless{}} \DecValTok{30} \SpecialCharTok{\textasciitilde{}} \StringTok{"Under 30"}\NormalTok{,}
\NormalTok{    Age }\SpecialCharTok{\textgreater{}=} \DecValTok{30} \SpecialCharTok{\&}\NormalTok{ Age }\SpecialCharTok{\textless{}} \DecValTok{40} \SpecialCharTok{\textasciitilde{}} \StringTok{"30{-}39"}\NormalTok{,}
\NormalTok{    Age }\SpecialCharTok{\textgreater{}=} \DecValTok{40} \SpecialCharTok{\&}\NormalTok{ Age }\SpecialCharTok{\textless{}} \DecValTok{50} \SpecialCharTok{\textasciitilde{}} \StringTok{"40{-}49"}\NormalTok{,}
\NormalTok{    Age }\SpecialCharTok{\textgreater{}=} \DecValTok{50} \SpecialCharTok{\textasciitilde{}} \StringTok{"50+"}
\NormalTok{  ))}
\end{Highlighting}
\end{Shaded}

\subsection{\texorpdfstring{\textbf{4. Age and
Burnout}}{4. Age and Burnout}}\label{age-and-burnout}

Burnout levels varied by age group. Younger employees (under 30) had the
highest proportion of burnout, while workers aged 40+ were less likely
to report severe burnout. Age bands were grouped as Under 30, 30--39,
40--49, and 50+ for clarity.

\begin{verbatim}
## # A tibble: 6 x 4
## # Groups:   Age_Group [2]
##   Age_Group Burnout_Level Count Percentage
##   <chr>     <chr>         <int>      <dbl>
## 1 30-39     High            256       36.7
## 2 30-39     Low             149       21.4
## 3 30-39     Medium          292       41.9
## 4 40-49     High            230       31.1
## 5 40-49     Low             180       24.4
## 6 40-49     Medium          329       44.5
\end{verbatim}

\begin{Shaded}
\begin{Highlighting}[]
\FunctionTok{ggplot}\NormalTok{(burnout\_by\_age, }
       \FunctionTok{aes}\NormalTok{(}\AttributeTok{x =}\NormalTok{ Age\_Group, }
           \AttributeTok{y =}\NormalTok{ Percentage, }
           \AttributeTok{fill =}\NormalTok{ Burnout\_Level)) }\SpecialCharTok{+}
  
  \FunctionTok{geom\_col}\NormalTok{(}\AttributeTok{width =} \FloatTok{0.6}\NormalTok{, }\AttributeTok{color =} \StringTok{"white"}\NormalTok{, }\AttributeTok{linewidth =} \FloatTok{0.3}\NormalTok{) }\SpecialCharTok{+}  \CommentTok{\# Clean white outlines}
  
  \FunctionTok{scale\_fill\_manual}\NormalTok{(}\AttributeTok{values =} \FunctionTok{c}\NormalTok{(}
    \StringTok{"Low"} \OtherTok{=} \StringTok{"\#7bc8a4"}\NormalTok{,   }
    \StringTok{"Medium"} \OtherTok{=} \StringTok{"\#ffb347"}\NormalTok{, }
    \StringTok{"High"} \OtherTok{=} \StringTok{"\#ff6b6b"}    
\NormalTok{  )) }\SpecialCharTok{+}
  
  \FunctionTok{labs}\NormalTok{(}
    \AttributeTok{x =} \StringTok{"Age Group"}\NormalTok{, }
    \AttributeTok{y =} \StringTok{"Percentage (\%)"}\NormalTok{,}
    \AttributeTok{fill =} \StringTok{"Burnout Level"}
\NormalTok{  ) }\SpecialCharTok{+}
  
  \FunctionTok{theme\_minimal}\NormalTok{(}\AttributeTok{base\_size =} \DecValTok{12}\NormalTok{) }\SpecialCharTok{+}
  \FunctionTok{theme}\NormalTok{(}
    \AttributeTok{legend.position =} \StringTok{"top"}\NormalTok{,}
    \AttributeTok{panel.grid.major.x =} \FunctionTok{element\_blank}\NormalTok{(),}
    \AttributeTok{plot.title =} \FunctionTok{element\_text}\NormalTok{(}\AttributeTok{hjust =} \FloatTok{0.5}\NormalTok{, }\AttributeTok{face =} \StringTok{"bold"}\NormalTok{)}
\NormalTok{  ) }\SpecialCharTok{+}
  
  \FunctionTok{geom\_text}\NormalTok{(}\FunctionTok{aes}\NormalTok{(}\AttributeTok{label =} \FunctionTok{paste0}\NormalTok{(}\FunctionTok{round}\NormalTok{(Percentage), }\StringTok{"\%"}\NormalTok{)),}
            \AttributeTok{position =} \FunctionTok{position\_stack}\NormalTok{(}\AttributeTok{vjust =} \FloatTok{0.5}\NormalTok{),}
            \AttributeTok{color =} \StringTok{"white"}\NormalTok{,}
            \AttributeTok{size =} \FloatTok{3.8}\NormalTok{)}
\end{Highlighting}
\end{Shaded}

\pandocbounded{\includegraphics[keepaspectratio]{Remote_Work_Analysis_files/figure-latex/unnamed-chunk-5-1.pdf}}

\section{\texorpdfstring{\textbf{Conclusion}}{Conclusion}}\label{conclusion}

The analysis highlights that remote workers and younger employees are
more at risk of high burnout. Additionally, mental health conditions
strongly correlate with increased burnout.

\section{**Recommendations:}\label{recommendations}

Based on the findings, organizations should consider the following
actions:

\begin{itemize}
\tightlist
\item
  Implement targeted wellness programs for remote employees, focusing on
  reducing screen fatigue.
\item
  Provide age-specific support initiatives, such as mentorship for
  younger staff.
\item
  Encourage flexible work arrangements for high-burnout groups.
\item
  Offer mental health resources, such as counseling and stress
  management workshops.
\item
  Monitor workload and ensure employees maintain work-life balance.
\end{itemize}

\end{document}
